\def\paperversiondraft{draft}
\def\paperversionblind{normal}
\def\paperversioncameraIEEE{cameraIEEE}

% If no draft paper-version is requested, compile in 'normal' mode
\ifx\paperversion\paperversiondraft
\else
  \ifx\paperversion\paperversioncameraIEEE
  \else
    \def\paperversion{normal}
  \fi
\fi

\def\grammarlyon{on}

% Disable review mode for grammarly version.
\ifx\grammarly\grammarlyon
\def\ClassReview{}
\else
\def\ClassReview{review,}
\fi

\ifx\paperversion\paperversioncameraIEEE
  \documentclass[conference]{IEEEtran}
\else
  \documentclass[\ClassReview anonymous, sigplan]{acmart}
\fi

\def\acmversionanonymous{anonymous}
\def\acmversionjournal{journal}
\def\acmversionnone{none}

\ifx\paperversion\paperversioncameraIEEE
    \def\acmversion{none}
\else
  \makeatletter
  \if@ACM@anonymous
    \def\acmversion{anonymous}
  \else
    \def\acmversion{journal}
  \fi
  \makeatother
\fi

\usepackage{colortbl}

% 'draftonly' environment
\usepackage{environ}
\ifx\paperversion\paperversiondraft
\newenvironment{draftonly}{}{}
\else
\NewEnviron{draftonly}{}
\fi

% Most PL conferences are edited by conference-publishing.com. Follow their
% advice to add the following packages.
%
% The first enables the use of UTF-8 as character encoding, which is the
% standard nowadays. The second ensures the use of font encodings that support
% accented characters etc. (Why should I use this?). The mictotype package
% enables certain features 'to­wards ty­po­graph­i­cal per­fec­tion
\usepackage[utf8]{inputenc}
\usepackage[T1]{fontenc}
\usepackage{microtype}

\usepackage{minted}
\usemintedstyle{colorful}
\newminted[mlir]{tools/MLIRLexer.py:MLIRLexerOnlyOps -x}{mathescape}

\usepackage{xargs}
\usepackage{lipsum}
\usepackage[textsize=tiny]{todonotes}
\usepackage{xparse}
\usepackage{xifthen, xstring}
\usepackage[normalem]{ulem}
\usepackage{xspace}
\usepackage{marginnote}
\usepackage{etoolbox}

\makeatletter
\font\uwavefont=lasyb10 scaled 652

\newcommand\colorwave[1][blue]{\bgroup\markoverwith{\lower3\p@\hbox{\uwavefont\textcolor{#1}{\char58}}}\ULon}
\makeatother

\ifx\paperversion\paperversiondraft
\newcommand\createtodoauthor[2]{%
\def\tmpdefault{emptystring}
\expandafter\newcommand\csname #1\endcsname[2][\tmpdefault]{\def\tmp{##1}\ifthenelse{\equal{\tmp}{\tmpdefault}}
   {\todo[linecolor=#2,backgroundcolor=#2,bordercolor=#2]{\textbf{#1:} ##2}}
   {\ifthenelse{\equal{##2}{}}{\colorwave[#2]{##1}\xspace}{\todo[linecolor=#2,backgroundcolor=#2,bordercolor=#2]{\textbf{#1:} ##2}\colorwave[#2]{##1}}}}
\expandafter\newcommand\csname #1f\endcsname[2][\tmpdefault]{
	\smash{\marginnote{
		\todo[inline,linecolor=#2,backgroundcolor=#2,bordercolor=#2]{\textbf{#1 (Figure):} ##2}}}
   }
}
%
\else
\newcommand\createtodoauthor[2]{%
\expandafter\newcommand\csname #1\endcsname[2][]{##1}%
\expandafter\newcommand\csname #1f\endcsname[2][]{##1}%
}%
\fi

% Broaden margins to make room for todo notes
\makeatletter
\patchcmd{\@addmarginpar}{\ifodd\c@page}{\ifodd\c@page\@tempcnta\m@ne}{}{}
\makeatother
\ifx\paperversion\paperversiondraft
  \ifx\acmversion\acmversionjournal
    \geometry{asymmetric}
    \paperwidth=\dimexpr \paperwidth + 3.5cm\relax
    \oddsidemargin=\dimexpr\oddsidemargin + 0cm\relax
    \evensidemargin=\dimexpr\evensidemargin + 0cm\relax
    \marginparwidth=\dimexpr \marginparwidth + 3cm\relax
    \setlength{\marginparwidth}{4.6cm}
    % This makeatletter box helps to move notes to the right
    \makeatletter
    \long\def\@mn@@@marginnote[#1]#2[#3]{%
      \begingroup
        \ifmmode\mn@strut\let\@tempa\mn@vadjust\else
          \if@inlabel\leavevmode\fi
          \ifhmode\mn@strut\let\@tempa\mn@vadjust\else\let\@tempa\mn@vlap\fi
        \fi
        \@tempa{%
          \vbox to\z@{%
            \vss
            \@mn@margintest
            \if@reversemargin\if@tempswa
                \@tempswafalse
              \else
                \@tempswatrue
            \fi\fi
            %\if@tempswa
              \rlap{%
                \if@mn@verbose
                  \PackageInfo{marginnote}{xpos seems to be \@mn@currxpos}%
                \fi
                \begingroup
                  \ifx\@mn@currxpos\relax\else\ifx\@mn@currxpos\@empty\else
                      \kern-\dimexpr\@mn@currxpos\relax
                  \fi\fi
                  \ifx\@mn@currpage\relax
                    \let\@mn@currpage\@ne
                  \fi
                  \if@twoside\ifodd\@mn@currpage\relax
                      \kern\oddsidemargin
                    \else
                      \kern\evensidemargin
                    \fi
                  \else
                    \kern\oddsidemargin
                  \fi
                  \kern 1in
                \endgroup
                \kern\marginnotetextwidth\kern\marginparsep
                \vbox to\z@{\kern\marginnotevadjust\kern #3
                  \vbox to\z@{%
                    \hsize\marginparwidth
                    \linewidth\hsize
                    \kern-\parskip
                    \marginfont\raggedrightmarginnote\strut\hspace{\z@}%
                    \ignorespaces#2\endgraf
                    \vss}%
                  \vss}%
              }%
          }%
        }%
      \endgroup
    }
    \makeatother
  \else
    \paperwidth=\dimexpr \paperwidth + 6cm\relax
    \oddsidemargin=\dimexpr\oddsidemargin + 3cm\relax
    \evensidemargin=\dimexpr\evensidemargin + 3cm\relax
    \marginparwidth=\dimexpr \marginparwidth + 3cm\relax
    \setlength{\marginparwidth}{4.6cm}
  \fi
\fi

% We use the following color scheme
% 
% This scheme is both print-friendly and colorblind safe for
% up to four colors (including the red tones makes it not
% colorblind safe any more)
%
% https://colorbrewer2.org/#type=qualitative&scheme=Paired&n=4

\definecolor{pairedNegOneLightGray}{HTML}{cacaca}
\definecolor{pairedNegTwoDarkGray}{HTML}{827b7b}
\definecolor{pairedOneLightBlue}{HTML}{a6cee3}
\definecolor{pairedTwoDarkBlue}{HTML}{1f78b4}
\definecolor{pairedThreeLightGreen}{HTML}{b2df8a}
\definecolor{pairedFourDarkGreen}{HTML}{33a02c}
\definecolor{pairedFiveLightRed}{HTML}{fb9a99}
\definecolor{pairedSixDarkRed}{HTML}{e31a1c}

\definecolor{mygreen}{rgb}{0,0.6,0}
%% Some colors for the polts (color blind safe)
%% https://www.acm.org/publications/proceedings-template
%% note RGB if you want to put the color in decimal.
%% color scheme: 3 colors
\definecolor{blind_safe_one_scheme_three_colors}{RGB}{102,194,165}
\definecolor{blind_safe_two_scheme_three_colors}{RGB}{252,141,98}
\definecolor{blind_safe_three_scheme_three_colors}{RGB}{141,160,203}
%% color scheme: 4 colors
\definecolor{blind_safe_one_scheme_four_colors}{RGB}{166,206,227}
\definecolor{blind_safe_two_scheme_four_colors}{RGB}{31,120,180} 
\definecolor{blind_safe_three_scheme_four_colors}{RGB}{178,223,138}
\definecolor{blind_safe_four_scheme_four_colors}{RGB}{51,160,44}
%% color scheme: 7 colors
\definecolor{blind_safe_one_scheme_seven_colors}{RGB}{118,42,131}
\definecolor{blind_safe_two_scheme_seven_colors}{RGB}{175,141,195}
\definecolor{blind_safe_three_scheme_seven_colors}{RGB}{231,212,232}
\definecolor{blind_safe_four_scheme_seven_colors}{RGB}{247,247,247}
\definecolor{blind_safe_five_scheme_seven_colors}{RGB}{217,240,211}
\definecolor{blind_safe_six_scheme_seven_colors}{RGB}{127,191,123}
\definecolor{blind_safe_seven_scheme_seven_colors}{RGB}{27,120,55}

\definecolor{butter1}{rgb}{0.988,0.914,0.310}
\definecolor{butter2}{rgb}{0.929,0.831,0.000}
\definecolor{butter3}{rgb}{0.769,0.627,0.000}

\definecolor{orange1}{rgb}{0.988,0.686,0.243}
\definecolor{orange2}{rgb}{0.961,0.475,0.000}
\definecolor{orange3}{rgb}{0.808,0.361,0.000}

\definecolor{chocolate1}{rgb}{0.914,0.725,0.431}
\definecolor{chocolate2}{rgb}{0.757,0.490,0.067}
\definecolor{chocolate3}{rgb}{0.561,0.349,0.008}

\definecolor{chameleon1}{rgb}{0.541,0.886,0.204}
\definecolor{chameleon2}{rgb}{0.451,0.824,0.086}
\definecolor{chameleon3}{rgb}{0.306,0.604,0.024}

\definecolor{skyblue1}{rgb}{0.447,0.624,0.812}
\definecolor{skyblue2}{rgb}{0.204,0.396,0.643}
\definecolor{skyblue3}{rgb}{0.125,0.290,0.529}

\definecolor{plum1}{rgb}{0.678,0.498,0.659}
\definecolor{plum2}{rgb}{0.459,0.314,0.482}
\definecolor{plum3}{rgb}{0.361,0.208,0.400}

\definecolor{scarletred1}{rgb}{0.937,0.161,0.161}
\definecolor{scarletred2}{rgb}{0.800,0.000,0.000}
\definecolor{scarletred3}{rgb}{0.643,0.000,0.000}

\definecolor{aluminium1}{rgb}{0.933,0.933,0.925}
\definecolor{aluminium2}{rgb}{0.827,0.843,0.812}
\definecolor{aluminium3}{rgb}{0.729,0.741,0.714}
\definecolor{aluminium4}{rgb}{0.533,0.541,0.522}
\definecolor{aluminium5}{rgb}{0.333,0.341,0.325}
\definecolor{aluminium6}{rgb}{0.180,0.204,0.212}

\createtodoauthor{grosser}{pairedOneLightBlue}
\createtodoauthor{authorTwo}{pairedTwoDarkBlue}
\createtodoauthor{authorThree}{pairedThreeLightGreen}
\createtodoauthor{authorFour}{pairedFourDarkGreen}
\createtodoauthor{authorFive}{pairedFiveLightRed}
\createtodoauthor{authorSix}{pairedSixDarkRed}

\graphicspath{{./images/}}

% Define macros that are used in this paper
%
% We require all macros to end with a delimiter (by default {}) to enusure
% that LaTeX adds whitespace correctly.
\makeatletter
\newcommand\requiredelimiter[2][########]{%
  \ifdefined#2%
    \def\@temp{\def#2#1}%
    \expandafter\@temp\expandafter{#2}%
  \else
    \@latex@error{\noexpand#2undefined}\@ehc
  \fi
}
\@onlypreamble\requiredelimiter
\makeatother

\newcommand\newdelimitedcommand[2]{
\expandafter\newcommand\csname #1\endcsname{#2}
\expandafter\requiredelimiter
\csname #1 \endcsname
}

\newdelimitedcommand{toolname}{Tool}


% Print \autoref as "Section X.Y.Z"
\ifx\paperversion\paperversioncameraIEEE
\else
  \renewcommand*{\sectionautorefname}{Section}
  \renewcommand*{\subsectionautorefname}{Section}
  \renewcommand*{\subsubsectionautorefname}{Section}
\fi

% \circled command to print a colored circle.
% \circled{1} pretty-prints "(1)"
% This is useful to refer to labels that are embedded within figures.
\usepackage{tikz}
\usetikzlibrary{arrows}
\usetikzlibrary{shapes}
\DeclareRobustCommand\circled[2][]{\ifthenelse{\isempty{#1}}{\tikz[baseline=(char.base)]{\node[shape=circle,fill=pairedOneLightBlue,inner sep=1pt] (char) {#2};}}{\autoref{#1}: \hyperref[#1]{\tikz[baseline=(char.base)]{\node[shape=circle,fill=pairedOneLightBlue,inner sep=1pt] (char) {#2};}}}}


\ifx\paperversion\paperversioncameraIEEE
\else
  \ifx\acmversion\acmversionjournal
  %% Journal information (used by PACMPL format)
  %% Supplied to authors by publisher for camera-ready submission
  \acmJournal{PACMPL}
  \acmVolume{1}
  \acmNumber{1}
  \acmArticle{1}
  \acmYear{2017}
  \acmMonth{1}
  \acmDOI{10.1145/nnnnnnn.nnnnnnn}
  \startPage{1}
  \else
  %% Conference information (used by SIGPLAN proceedings format)
  %% Supplied to authors by publisher for camera-ready submission
  \acmConference[PL'17]{ACM SIGPLAN Conference on Programming Languages}{January 01--03, 2017}{New York, NY, USA}
  \acmYear{2017}
  \acmISBN{978-x-xxxx-xxxx-x/YY/MM}
  \acmDOI{10.1145/nnnnnnn.nnnnnnn}
  \startPage{1}
  \fi
\fi


%% Copyright information
%% Supplied to authors (based on authors' rights management selection;
%% see authors.acm.org) by publisher for camera-ready submission
%\setcopyright{none}             %% For review submission
%\setcopyright{acmcopyright}
%\setcopyright{acmlicensed}
%\setcopyright{rightsretained}
%\copyrightyear{2017}           %% If different from \acmYear


%% Bibliography style
\ifx\paperversion\paperversioncameraIEEE
  \bibliographystyle{IEEETran}
\else
  \bibliographystyle{ACM-Reference-Format}
\fi
%% Citation style
%% Note: author/year citations are required for papers published as an
%% issue of PACMPL.
%\citestyle{acmauthoryear}  %% For author/year citations
%\citestyle{acmnumeric}     %% For numeric citations
%\setcitestyle{nosort}      %% With 'acmnumeric', to disable automatic
                            %% sorting of references within a single citation;
                            %% e.g., \cite{Smith99,Carpenter05,Baker12}
                            %% rendered as [14,5,2] rather than [2,5,14].
%\setcitesyle{nocompress}   %% With 'acmnumeric', to disable automatic
                            %% compression of sequential references within a
                            %% single citation;
                            %% e.g., \cite{Baker12,Baker14,Baker16}
                            %% rendered as [2,3,4] rather than [2-4].



\begin{document}

%% Title information
\ifx\paperversion\paperversioncameraIEEE
  \title{Full title}
\else
  \title[]{PyLinnea}       %% [Short Title] is optional;
                                        %% when present, will be used in
                                        %% header instead of Full Title.
  \subtitle{Compiler Support for Linear-algebra Computations in MLIR} %% \subtitle is optional
\fi


%% Author information
%% Contents and number of authors suppressed with 'anonymous'.
%% Each author should be introduced by \author, followed by
%% \authornote (optional), \orcid (optional), \affiliation, and
%% \email.
%% An author may have multiple affiliations and/or emails; repeat the
%% appropriate command.
%% Many elements are not rendered, but should be provided for metadata
%% extraction tools.

%% Author with single affiliation.
\ifx\paperversion\paperversioncameraIEEE
  \author{\IEEEauthorblockN{1\textsuperscript{st} Given Name Surname}
  \IEEEauthorblockA{\textit{dept. name of organization (of Aff.)} \\
  \textit{name of organization (of Aff.)}\\
  City, Country \\
  email address or ORCID}
  \and
  \IEEEauthorblockN{2\textsuperscript{nd} Given Name Surname}
  \IEEEauthorblockA{\textit{dept. name of organization (of Aff.)} \\
  \textit{name of organization (of Aff.)}\\
  City, Country \\
  email address or ORCID}
  \and
  \IEEEauthorblockN{3\textsuperscript{rd} Given Name Surname}
  \IEEEauthorblockA{\textit{dept. name of organization (of Aff.)} \\
  \textit{name of organization (of Aff.)}\\
  City, Country \\
  email address or ORCID}
  \and
  \IEEEauthorblockN{4\textsuperscript{th} Given Name Surname}
  \IEEEauthorblockA{\textit{dept. name of organization (of Aff.)} \\
  \textit{name of organization (of Aff.)}\\
  City, Country \\
  email address or ORCID}
  \and
  \IEEEauthorblockN{5\textsuperscript{th} Given Name Surname}
  \IEEEauthorblockA{\textit{dept. name of organization (of Aff.)} \\
  \textit{name of organization (of Aff.)}\\
  City, Country \\
  email address or ORCID}
  \and
  \IEEEauthorblockN{6\textsuperscript{th} Given Name Surname}
  \IEEEauthorblockA{\textit{dept. name of organization (of Aff.)} \\
  \textit{name of organization (of Aff.)}\\
  City, Country \\
  email address or ORCID}
  }
\else
  \author{First1 Last1}
  \authornote{with author1 note}          %% \authornote is optional;
                                        %% can be repeated if necessary
  \orcid{nnnn-nnnn-nnnn-nnnn}             %% \orcid is optional
  \affiliation{
    \position{Position1}
    \department{Department1}              %% \department is recommended
    \institution{Institution1}            %% \institution is required
    \streetaddress{Street1 Address1}
    \city{City1}
    \state{State1}
    \postcode{Post-Code1}
    \country{Country1}
  }
  \email{first1.last1@inst1.edu}          %% \email is recommended

  \author{First2 Last2}
  \authornote{with author2 note}          %% \authornote is optional;
                                        %% can be repeated if necessary
  \orcid{nnnn-nnnn-nnnn-nnnn}             %% \orcid is optional
  \affiliation{
    \position{Position2a}
    \department{Department2a}             %% \department is recommended
    \institution{Institution2a}           %% \institution is required
    \streetaddress{Street2a Address2a}
    \city{City2a}
    \state{State2a}
    \postcode{Post-Code2a}
    \country{Country2a}
  }
  \email{first2.last2@inst2a.com}         %% \email is recommended
  \affiliation{
    \position{Position2b}
    \department{Department2b}             %% \department is recommended
    \institution{Institution2b}           %% \institution is required
    \streetaddress{Street3b Address2b}
    \city{City2b}
    \state{State2b}
    \postcode{Post-Code2b}
    \country{Country2b}
  }
  \email{first2.last2@inst2b.org}         %% \email is recommended
\fi

\def\makeabstract{
\begin{abstract}
% An abstract should consist of six main sentences:
%  1. Introduction. In one sentence, what’s the topic?
%  2. State the problem you tackle.
%  3. Summarize (in one sentence) why nobody else has adequately answered the research question yet.
%  4. Explain, in one sentence, how you tackled the research question.
%  5. In one sentence, how did you go about doing the research that follows from your big idea.
%  6. As a single sentence, what’s the key impact of your research?

% (http://www.easterbrook.ca/steve/2010/01/how-to-write-a-scientific-abstract-in-six-easy-steps/)

  Dense linear algebra arises in numerical software in computational science
  and engineering. Often computations in dense linear algebra rely on
  highly-optimized building blocks from libraries such as BLAS and LAPACK.
  While such libraries provide portable performance for a wide range of
  computing architectures, they still present limitations in terms of
  flexibility. Therefore, we advocate for a compiler where we can reason about
  algebra computation and allow us to generate code automatically from a
  definition of the computation. This paper disucess integrating this idea in MLIR.


%\lipsum[1]
\end{abstract}
}

\ifx\paperversion\paperversioncameraIEEE
\else
  \makeabstract
\fi

% Only add ACM notes and keywords in camera ready version
% Drop citations and footnotes in draft and blind mode.
\ifx\acmversion\acmversionanonymous
\settopmatter{printacmref=false} % Removes citation information below abstract
\renewcommand\footnotetextcopyrightpermission[1]{} % removes footnote with conference information in first column
\fi
\ifx\acmversion\acmversionjournal
%% 2012 ACM Computing Classification System (CSS) concepts
%% Generate at 'http://dl.acm.org/ccs/ccs.cfm'.
\begin{CCSXML}
<ccs2012>
<concept>
<concept_id>10011007.10011006.10011008</concept_id>
<concept_desc>Software and its engineering~General programming languages</concept_desc>
<concept_significance>500</concept_significance>
</concept>
<concept>
<concept_id>10003456.10003457.10003521.10003525</concept_id>
<concept_desc>Social and professional topics~History of programming languages</concept_desc>
<concept_significance>300</concept_significance>
</concept>
</ccs2012>
\end{CCSXML}

\ccsdesc[500]{Software and its engineering~General programming languages}
\ccsdesc[300]{Social and professional topics~History of programming languages}
%% End of generated code

%% Keywords
%% comma separated list
\keywords{keyword1, keyword2, keyword3} 
\fi

%% \maketitle
%% Note: \maketitle command must come after title commands, author
%% commands, abstract environment, Computing Classification System
%% environment and commands, and keywords command.
\maketitle
\ifx\grammarly\grammarlyon 
\onecolumn 
\else 
\fi

% IEEE wants abstract and keywords 
% after maketitle
\ifx\paperversion\paperversioncameraIEEE
  \makeabstract
%% Keywords
%% comma separated list
  \begin{IEEEkeywords}
    keyword1, keyword2, keyword3  %% \keywords is optional
  \end{IEEEkeywords}
\fi

\section{Introduction}

A significant part of processor time is spent on mathematical algorithms used
in simulations, machine learning, communication, signal processing, computer
vision, and other domains. Both scenarios need fast code. The mathematics used
in these domains may differ widely. Still, the actual computations often fall
into the realm of linear algebra, meaning sequences of computations on matrices
and vectors. Optimizing these sequences of operation thus becomes crucial.

\begin{itemize}
  \item No idea what is today landscape in optimizing dense-linear algebra.
\end{itemize}

\begin{figure}
% Link to figure
%
% https://docs.google.com/drawings/d/1juKp43D3rLC-luBQPwQZ_wCnDK2S_6C1k6USV0wKE0g/edit?usp=sharing
\includegraphics[width=0.7\columnwidth]{images/Impact2022.drawio.pdf}
\caption{The PyLinnea compiler for dense-linear algebra.}
\end{figure}

~\\
Our contributions are:

\begin{itemize}
	\item An IR representation to model linear-algebra operations, types and properties.
	\item A lowering path from such representation to LLVM IR.
  \item A python frontend to express linear algebra problems.
\end{itemize}

We call our tool PyLinnea and it's built on top of MLIR.

\section{The Linnea Dialect}

Dense linear-algebra support is captured in MLIR by introducing a new Linnea
Dialect, which provides attributes, types, operations, and transformations
required to make dense computation as first-class citizens in the compiler IR.
Linnea forms a bridge between the high-level dense linear-algebra mathematics
expressed in Python DSL and a more BLAS-like dialect: Linalg. From Linalg we lower
to LLVM IR and then machine code.

\paragraph{Matrix Attribute}

\begin{listing}[]
\begin{center}
\begin{minipage}[]{0.5\textwidth}
\begin{minted}[fontsize=\scriptsize, escapeinside=@@]{cpp}
  @\color{orange3}{#linnea.property}@<[@\color{skyblue3}{'lowerTri'}@]>
\end{minted}
\end{minipage}
\caption{Linnea attributes attach properties to the underneath algebraic object.}
\label{lst:attribute}
\end{center}
\end{listing}

Central to the design of our dialect is the \texttt{Matrix Attribute} a
notation mechanism -- inspired from the Sparse Tensor Dialect -- to encode
compile-time information that defines a given \texttt{Matrix} type. A
\texttt{Matrix Attribute} is thus a list of non-conflicting properties that
define the underneath algebraic object (i.e., lower or upper triangular).
Listing~\ref{lst:attribute} shows an attribute to flag an algebraic type as
lower triangular. During bufferization~\footnote{bufferization materializes
memref (i.e., malloc) from tensors}, the underneath algebraic object is
materialized using a straightforward \texttt{memref}. The attributes dictates
how the memory is filled (i.e., set to zeros the element below the diagonal for
an upper triangular matrix). In the future, we would like to depart from
\texttt{memref} and use more appropriate containers.

\paragraph{Types: Matrix, Term and Identity}

Linnea provides three built-in types: Matrix, Term and Identity. Matrix
represents a specialization of an n-dimensional tensor type that guarantees a
2-dimensional shape. It comes with an attribute that describes a set of
non-conflicting properties, a static dimension, and an MLIR built-in type that
describes each stored element's type (i.e., f32). Listing~\ref{lst:type} shows
how a 5 $\times$ 5 lower-triangular matrix can be represented. A term type
represents the result of Linnea \texttt{equationOp}. It is a generic type,
meaning that it will be replaced by a concrete one (i.e., MatrixType) (only after
optimization and simplification the result of an \texttt{equationOp} is known).
Finally, an Identity type represents the identity matrix -- a square matrix
where all the elements on the diagonal are 1. The identity types can unlock
some interesting optimizations, for example: $A * I = A$ or $I * I = I$.

\begin{listing}[]
\begin{center}
\begin{minipage}[]{0.5\textwidth}
\begin{minted}[fontsize=\scriptsize, escapeinside=@@]{cpp}
  !linnea.matrix@\color{orange3}{#linnea.property}@<[@\color{skyblue3}{'lowerTri'}@, [5, 5], @\color{red}{f32}@]>
\end{minted}
\end{minipage}
  \caption{A type to represent a 5 $\times$ 5 lower-triangular matrix with \texttt{f32} as element type.}
\label{lst:type}
\end{center}
\end{listing}

\paragraph{Operations}

Table shows the operations exposed by the Linnea
dialect~\ref{table:operations}. \texttt{Init} and \texttt{Fill} initialize and
fill an algebraic object, respectively. \texttt{Mul} and \texttt{Add} are
variadic operations that takes as input an arbitrary numbers of Values of
Linnea types.  The \texttt{Equation} represents a ``bag'' of Linnea operations
that together represent a given mathematical expression. We build a symbolic
representation of the program by walking each Equation starting from the yield
operation, which gets simplified and rematerialized to new low-level Linnea IR.
For example, variadic multiplications are  rewritten as binary multiplications
with an optimal parenthesization. 

\begin{table}
\begin{center}
\begin{tabular}{ll}
    \toprule
    \footnotesize{Op name}     & \footnotesize{Description} \\ \midrule
    \footnotesize{Init} & \footnotesize{Materialize the algebraic object using a \texttt{memref}} \\
    \rowcolor{aluminium1}
    \footnotesize{Fill} & \footnotesize{Initialize the algebraic object according to its properties}  \\
    \rowcolor{aluminium1}
    \footnotesize{Equation} & \footnotesize{Represents a linear-algebra equation} \\
    \footnotesize{Add} & \footnotesize{Variadic addition of different algebraic objects} \\
    \rowcolor{aluminium1}
    \footnotesize{Yield} & \footnotesize{Return the result of an Equation} \\
    \footnotesize{Mul} & \footnotesize{Variadic multiplication of different algebraic objects} \\ \bottomrule
\end{tabular}
\end{center}
\caption{Operations exposed by the Linnea dialect.}
\label{table:operations}
\end{table} 

\paragraph{Transformations}

Currently, we limit ourselves to three major
transformations: matrix-chain reordering, identity simplification and
properties propagation. Matrix chain reordering is an algorithmic improvement
that minimizes the number of scalar multiplications when multiplying a chain of
matrices. We implement the algorithm described in~\cite{cormen2009introduction}
and generalize it to account for matrix properties as
in~\cite{barthels2020automatic}. Identity simplification consists of exploiting
the identity matrix to simplify the computation. For example, $A * I
\rightarrow A$.  Finally, Linnea's IR makes it possible to annotate matrices
with properties. But, it is also essential to understand the properties of
intermediate results as the computation unfold. Thus we introduce a symbolic
engine and encode a set of inference rules such as $lowerTrinagular(A)
\rightarrow upperTriangular(A^T)$. We use the symbolic engine to replace a
\texttt{TermType} with a concrete type.

\section{PyLinnea Compiler Usage}

PyLinnea provides users with a convenient Python DSL language to express their
computation. It does not require any knowledge about the internal intermediate
representation or the different compiler passes involved in lowering a Python
specification to binary. For example, listing~\ref{lst:python} shows how a user
can specify a multiplication between two lower triangular matrices. 

\begin{listing}[]
\begin{center}
\begin{minipage}[]{0.5\textwidth}
\begin{minted}[fontsize=\scriptsize, linenos, escapeinside=@@]{cpp}
  n = 5
  m = 5
  Matrix A(n, m) <LowerTriangular>
  A = 23
  Matrix B(n, m) <LowerTrinagular>
  B = 26
  Matrix C(n, m) <>
  C = A * B
  print(C)
\end{minted}
\end{minipage}
  \caption{PyLinnea specification for a triangular matrix multiplication.}
\label{lst:python}
\end{center}
\end{listing}

Behind the scene, PyLinnea lowers the Python specification to the Linnea
dialect (see Listing~\ref{lst:endtoend}).  Line 3 and 4 map to the
initialization and fill operations both for the A and B matrix. Line 8 maps to
the \texttt{linnea.equation} operation. 

\begin{listing}[]
\begin{center}
\begin{minipage}[]{0.5\textwidth}
\begin{minted}[fontsize=\scriptsize, escapeinside=@@]{cpp}
  // initialize matrix A
  @\color{orange3}{%A}@ = linnea.init [5, 5] : 
    !linnea.matrix@\color{orange3}{#linnea.property}@<[@\color{skyblue3}{'lowerTri'}@, [5, 5], @\color{red}{f32}@]>
  %Af = linnea.fill(%fc, %A) : 
    @\color{red}{f32}@, !linnea.matrix@\color{orange3}{#linnea.property}@<[@\color{skyblue3}{'lowerTri'}@, [5, 5], @\color{red}{f32}@]>
  // initialize matrix B
  %B = linnea.init [5, 5] : 
    !linnea.matrix@\color{orange3}{#linnea.property}@<[@\color{skyblue3}{'lowerTri'}@, [5, 5], @\color{red}{f32}@]>
  %Bf = linnea.fill(%fc, %B) : 
    @\color{red}{f32}@, !linnea.matrix@\color{orange3}{#linnea.property}@<[@\color{skyblue3}{'lowerTri'}@, [5, 5], @\color{red}{f32}@]>
  // multiply the two and print the result
  @\color{orange3}{%0}@ = linnea.equation {
    @\color{orange3}{%1}@ = linnea.mul @\color{orange3}{%Af}@, @\color{orange3}{%Bf}@ :
      !linnea.matrix@\color{orange3}{#linnea.property}@<[@\color{skyblue3}{'lowerTri'}@, [5, 5], @\color{red}{f32}@]>
      !linnea.matrix@\color{orange3}{#linnea.property}@<[@\color{skyblue3}{'lowerTri'}@, [5, 5], @\color{red}{f32}@]> 
      -> !linnea.term
    linnea.yield @\color{orange3}{%1}@ : !linnea.term
  }
  linnea.print %0 : !linnea.term
\end{minted}
\end{minipage}
  \caption{A Linnea IR representation for a multiplication between two lower triangular matrices.}
\label{lst:endtoend}
\end{center}
\end{listing}

\section{Results}
\begin{itemize}
  \item chain results intel and amd
\end{itemize}

%\section{Related Work}

\section{Conclusion}

We presented PyLinnea, an end-to-end flow for dense-linear algebra operations
based on MLIR. Contrary to existing tools, PyLinnea is not limited by the
rigidity of current BLAS and LAPACK libraries. Still, more research is needed
to develop a proper code generation infrastructure, and more building blocks
are needed into the MLIR compiler infrastructure to make PyLinnea scalable and
flexible. Nevertheless, our research is a small step toward addressing the
limitation of today's dense-linear algebra optimizers and rethinking their
optimization flow in a more compiler-structured approach.

\ifx\paperversion\paperversioncameraIEEE
\else
%% Acknowledgments
\begin{acks}                            %% acks environment is optional
                                        %% contents suppressed with 'anonymous'
  %% Commands \grantsponsor{<sponsorID>}{<name>}{<url>} and
  %% \grantnum[<url>]{<sponsorID>}{<number>} should be used to
  %% acknowledge financial support and will be used by metadata
  %% extraction tools.
  This material is based upon work supported by the
  \grantsponsor{GS100000001}{National Science
    Foundation}{http://dx.doi.org/10.13039/100000001} under Grant
  No.~\grantnum{GS100000001}{nnnnnnn} and Grant
  No.~\grantnum{GS100000001}{mmmmmmm}.  Any opinions, findings, and
  conclusions or recommendations expressed in this material are those
  of the author and do not necessarily reflect the views of the
  National Science Foundation.
\end{acks}
\fi

%% Bibliography
\bibliography{references}


%% Appendix
\newpage
\appendix
\begin{draftonly}
\section{Formatting and Writing Guidelines}

These formatting guidelines aim to standardize our writing. They ensure that
papers with multiple authors have a consistent look and that commonly occurring
items are formatted in ways that are known to work well.

\subsection{Figures}

\paragraph{Referencing Figures} When referencing figures from the text we
ensure the following:
\begin{description}
      \item [All figures are referenced] A paper with un-referenced figures
	      appears incomplete.
      \item [References to figures are brief and easy to skip]~\\
                We minimize the number of words needed to refer to a figure. Reducing
                the number of non-information-carrying words directly increases
		the density of interesting content. When skipping references
		becomes easy, reading quickly while ignoring figures remains a
		smooth experience. The best and briefest reference to a figure
		is a link in parenthesis that is added after the subject
		representing the content depicted in a figure:\\
		{\color{pairedTwoDarkBlue}\textit{Figure
		X shows the design of A, which consists of ...}}\\
		$\to$ {\color{pairedFourDarkGreen}
		\textit{The design of A (Figure X) consists of ...}}
      \item [The text is always self-contained without figures] ~\\ The reader
                should be able to read the text without ever looking at any
                figure. They should still understand the text and get the key
		message of each figure directly from the text. By not forcing
		the reader to analyze a figure while reading, we increase
		readability as the reader can continue reading without having
		to skip between text and figures. Such writing style also helps
		to guide the thoughts of the reader, who can (for a moment)
		trust our summary of the figure and does not need to develop
		their own interpretation on-the-fly, a task which often yields
		results that do not fit the flow of our exposition. Readers
		typically only feel that their reading is interrupted if there
		is no explanation of a figure at all. Hence, we do not need to
		discuss all details of a figure, but half a sentence that explains the
		core idea is typically sufficient for a reader to continue
		reading.  By making
		our text self-contained even when ignoring figures the reader
		experiences a smooth and uninterrupted reading experience.\\
		{\color{pairedTwoDarkBlue}
		\textit{The speedups are presented in Figure X. < a new topic> }}\\
		$\to$ {\color{pairedFourDarkGreen}\textit{Our approach outperforms the state of the art
		XXX-library (Figure 3) demonstrating more than 4x speedup on
		test case 1 and 2 and a geometric mean speedup of 1.5x over all
		20 test cases.}}
\end{description}
We reference figures in text using
\texttt{\symbol{92}autoref\{fig:speedup\}} for a figure with label
\texttt{fig:speedup}.  The use of autoref ensures that all references
to figures are formatted consistently, e.g. as \autoref{fig:speedup}.

\paragraph{Color Scheme} 

In Figures we use a color scheme that is print-friendly and also visible
with red-green blindness. The following colors are all print-friendly
and red-green save when only using Color 1-4:

\medskip
{
	\small
\newcolumntype{a}{>{\columncolor{pairedOneLightBlue}}c}
\newcolumntype{b}{>{\columncolor{pairedTwoDarkBlue}}c}
\newcolumntype{d}{>{\columncolor{pairedThreeLightGreen}}c}
\newcolumntype{e}{>{\columncolor{pairedFourDarkGreen}}c}
\newcolumntype{f}{>{\columncolor{pairedFiveLightRed}}c}
\newcolumntype{g}{>{\columncolor{pairedSixDarkRed}}c}

\begin{tabular}{a b d e f g}
Color 1 & Color 2 & Color 3 & Color 4 & Color 5 & Color 6\\
\#a6cee3 & \#1f78b4 & \#b2df8a & \#33a02c & \#fb9a99 & \#e31a1c
\end{tabular}
}

We de-emphasize components in figures by using additionally two shades of gray.
Especially in complex figures, it is often helpful to de-emphasize visual
elements that we want to represent but that should not be the focus of a
reader's attention.

\medskip
{
	\small
\newcolumntype{h}{>{\columncolor{pairedNegOneLightGray}}c}
\newcolumntype{i}{>{\columncolor{pairedNegTwoDarkGray}}c}

\begin{tabular}{h i}
Color -1 & Color -2\\
\#cacaca & \#827b7b\\
\end{tabular}
}

Single-color graphs are plotted in Color 1 - Light Blue.

\paragraph{Labels in Figures}
Complex diagrams often benefit from labels inside the diagrams. We suggest to
use a filled circle (e.g, in light blue) to highlight these numbers and use
these references, e.g., \circled{1} implemented as \texttt{\textbackslash{}circled\{1\}}, in the text to refer to them.

\subsubsection{Plots} We use matplotlib to create performance
plots such as \autoref{fig:speedup}. We use the following
formatting guidelines:
\begin{itemize}
  \item Use a vertical y-label to make it easier to read.
  \item Remove top and right frames to reduce visual noise
	and allow the reader to focus on the data in the
	figure.
  \item Provide the concrete data at the top of each bar.
\end{itemize}

\noindent
We also suggest to follow these technical remarks:
\begin{itemize}
  \item Create pdf plots and do not use bitmap formats (e.g., png) to
	ensure high quality when zooming in.
  \item Avoid Type-3 bitmap fonts by
	setting fonttype to 42.
\end{itemize}

\begin{figure}
\includegraphics[width=\columnwidth]{plots/speedup}
\caption{Improved running speed after 4 weeks of training.
}
\label{fig:speedup}
\end{figure}

\subsubsection{Listings} We aim to use minted to create listings as much as
possible, as this allows us to edit code quickly. We use syntax highlighting
to make the parts of the code that matter most stand out. Hence, we keep
most code black, comments gray, and highlight just the MLIR operands that
we care about most.

\begin{listing}
% We cannot put '{' on a line after % in draftonly mode, as the hack we used to
% not include the draft section will interpret the listing as normal
% latex where '%' is a comment and {} need to match, which they will
% not if only one is commented.
\begin{mlir}
// This is a comment
def @foo(%0 : !dialect.type)
{
  %a = dialect.op(%0) : !dialect.type // $\color{black}\circled{a}$
}
\end{mlir}
\caption{A simple MLIR code example with markers. Markers can also be placed in
	captions and refer to labels, e.g. \circled[lst:example]{a}.}
\label{lst:example}
\end{listing}

\end{draftonly}


\end{document}
